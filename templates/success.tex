<% extends "base.tex" %>
<% block body %>
\begin{center}
\begin{tabular}{|c|c|}
  \hline
  \cellcolor[gray]{0.8}
  \begin{minipage}{5cm}
    \centering
    Classical Logic
  \end{minipage}
  &
  \cellcolor[gray]{0.8}
  \begin{minipage}{5cm}
    \centering
    Intuitionistic Logic
  \end{minipage}
  \\\hline

  {
    \huge
<%- if cl_result.is_provable() %>
    \cmark
    Provable
<%- else if cl_result.is_not_provable() %>
    \xmark
    Not provable
<%- else %>
    Unknown
<%- endif %>
  }
  &
  {
    \huge
<%- if int_result.is_provable() %>
    \cmark
    Provable
<%- else if int_result.is_not_provable() %>
    \xmark
    Not provable
<%- else %>
    Unknown
<%- endif %>
  }
  \\\hline
\end{tabular}
\end{center}

\section*{Input}

$<= prop %>$

<% match cl_result %>
<%- when SolverResult::Provable with (pf) %>
<%- match pf %>
<%- when Some with (pf) -%>
\section*{Proof (classical logic)}

The proposition is provable in classical logic. Below is one such proof diagram.

<% for fragment in pf %>
<%- if pf.len() > 1 %>
\subsection*{<= fragment.name %>}
<%- endif %>
\begin{prooftree}
<= fragment.source -%>
\end{prooftree}
<% endfor %>

<%- when None %>
<%- endmatch %>

<%- when SolverResult::NotProvable with (rft) %>
<%- match rft %>
<%- when Some with (rft) -%>
\section*{Refutation (classical logic)}

The proposition isn't provable in classical logic, thus neither in intuitionistic one.

You can confirm non-provability via the following refutation in boolean semantics.

<= rft.assignment %>
<%- when None -%>
<%- endmatch -%>
<%- when SolverResult::Unknown -%>
<%- endmatch -%>
<% match int_result %>
<%- when SolverResult::Provable with (pf) %>
<%- match pf %>
<%- when Some with (pf) -%>
\section*{Proof (intuitionistic logic)}

The proposition is provable in intuitionistic logic, thus provable too in classical one. Below is one such proof diagram.

<% for fragment in pf %>
<%- if pf.len() > 1 %>
\subsection*{<= fragment.name %>}
<%- endif %>
\begin{prooftree}
<= fragment.source -%>
\end{prooftree}
<% endfor %>

<%- when None %>
<%- endmatch %>

<%- when SolverResult::NotProvable with (rft) %>
<%- match rft %>
<%- when Some with (rft) -%>
\section*{Refutation (intuitionistic logic)}

The proposition isn't provable in intuitionistic logic.

You can confirm non-provability via the following refutation in Kripke semantics.

<= rft.frame %>
<= rft.assignment %>
<%- when None -%>
<%- endmatch -%>
<%- when SolverResult::Unknown -%>
<%- endmatch -%>

<%- endblock -%>
